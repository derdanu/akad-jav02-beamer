%
% Erstellt von Daniel Falkner
% bmw@740i.de
% 
\documentclass[xcolor=dvipsnames]{beamer}
\usepackage[T1]{fontenc}
\usepackage[utf8]{inputenc}
\usepackage[ngerman]{isodate}
\usepackage[justification=centering,figurename=Abb.]{caption}
\usepackage{listings}
\usepackage{color}


\definecolor{mygreen}{rgb}{0,0.6,0}
\definecolor{mygray}{rgb}{0.5,0.5,0.5}
\definecolor{mymauve}{rgb}{0.58,0,0.82}

\lstdefinelanguage{JavaScript}{
  keywords={break, case, catch, continue, debugger, default, delete, do, else, finally, for, function, if, in, instanceof, new, return, switch, this, throw, try, typeof, var, void, while, with},
  morecomment=[l]{//},
  morecomment=[s]{/*}{*/},
  morestring=[b]',
  morestring=[b]",
  sensitive=true
}

\lstdefinelanguage{CSS}
{morekeywords={color,background,margin, font, width, float, height, padding, box, opacity, border, right, top, shadow, radius, bottom},
sensitive=false,
morecomment=[l]{//},
morecomment=[s]{/*}{*/},
morestring=[b]",
} 

\lstset{ %
  backgroundcolor=\color{white},   % choose the background color; you must add \usepackage{color} or \usepackage{xcolor}
  basicstyle=\footnotesize,        % the size of the fonts that are used for the code
  breakatwhitespace=false,         % sets if automatic breaks should only happen at whitespace
  breaklines=true,                 % sets automatic line breaking
  captionpos=b,                    % sets the caption-position to bottom
  commentstyle=\color{mygreen},    % comment style
  deletekeywords={...},            % if you want to delete keywords from the given language
  escapeinside={\%*}{*)},          % if you want to add LaTeX within your code
  extendedchars=true,              % lets you use non-ASCII characters; for 8-bits encodings only, does not work with UTF-8
 % frame=single,                    % adds a frame around the code
  keepspaces=true,                 % keeps spaces in text, useful for keeping indentation of code (possibly needs columns=flexible)
  keywordstyle=\color{blue},       % keyword style
  language=Octave,                 % the language of the code
  morekeywords={*,...},            % if you want to add more keywords to the set
  numbers=left,                    % where to put the line-numbers; possible values are (none, left, right)
  numbersep=5pt,                   % how far the line-numbers are from the code
  numberstyle=\tiny\color{mygray}, % the style that is used for the line-numbers
  rulecolor=\color{black},         % if not set, the frame-color may be changed on line-breaks within not-black text (e.g. comments (green here))
  showspaces=false,                % show spaces everywhere adding particular underscores; it overrides 'showstringspaces'
  showstringspaces=false,          % underline spaces within strings only
  showtabs=true,                  % show tabs within strings adding particular underscores
  stepnumber=1,                    % the step between two line-numbers. If it's 1, each line will be numbered
  stringstyle=\color{mymauve},     % string literal style
  tabsize=2,                       % sets default tabsize to 2 spaces
  title=\lstname,                   % show the filename of files included with \lstinputlisting; also try caption instead of title
  belowskip= 0pt 
}

\usetheme{Frankfurt}
\usecolortheme[named=OliveGreen]{structure}
\renewcommand\thempfootnote{\arabic{mpfootnote}}

\newcommand*{\Title}{Grafische Programmierung mit Java} %Titel
\subtitle{Modul JAV02} %Untertitel
\newcommand*{\Author}{Daniel Falkner} %Name
\institute{AKAD University} %Uni
\titlegraphic{\includegraphics[scale=0.2]{akad_logo.png}} %Logo

\title{\Title}
\author{\Author}
\date{05/06.September.2014}

%Pdf Metainformationen
\subject{\Title}
\keywords{}

\begin{document}

%Titelseite
\begin{frame}
    \titlepage
\end{frame}

%Logo auf allen weiteren Folien
%\logo{\includegraphics[scale=0.1]{akad_logo.png}}

%Inhaltsverzeichniss


\section{Über mich}
\begin{frame} %%Eine Folie
  \frametitle{Über mich} %%Folientitel
  \framesubtitle{Daniel Falkner} %%Fielenuntertitel
  \begin{block}{}
	  \begin{itemize}
	  	\item AKAD Student - Bachelor of Science (Wirtschaftsinformatik)
  		\item T-Systems International GmbH - Telekom IT
  		\item IT-Architekt - System Analyst
		\item Projektleiter
		\item Proof of Concept Engineer
  		\item Debian Linux Administrator
	  \end{itemize}
  \end{block}
\end{frame}


\section{Java}
\begin{frame} %%Eine Folie
  \frametitle{Modul JAV02} %%Folientitel
  \framesubtitle{Aufgabe 2} %%Fielenuntertitel
  \begin{block}{Aufgabe 2}
	  \begin{itemize}
		\item Erläutern Sie ausführlich die elementaren Steuerelemente von Swing wie JLabel, JButton, JComboBox usw.
	  \end{itemize}

  \end{block}
\end{frame}


\subsection{JLabel}
\begin{frame}[fragile, shrink] %%Eine Folie
  \frametitle{JLabel} %%Folientitel
  \begin{block}{statischer Text, nicht editierbar}
	  \begin{itemize}
		\item reiner Text
	  \end{itemize}
  \end{block}

\begin{lstlisting}[language=java,basicstyle=\scriptsize\ttfamily]
// Wir holen uns ein Icon aus dem dem Java Look and Feel
        Icon icon = MetalIconFactory.getFileChooserHomeFolderIcon();
 
        // Wir erstellen ein JLabel mit einem Text und unserem Icon
        // Die horizontale Ausrichtung setzen wir auf "CENTER"
        JLabel label = new JLabel ("My 127.0.0.1 is my castle", 
            icon, JLabel.CENTER);
 
        // Die vertikale Ausrichtung des JLabels setzen wir auf "TOP"
        label.setVerticalAlignment(JLabel.TOP);
 
        // Die relative Ausrichtung des Textes zum Icon setzen wir auf "LEFT"
        label.setHorizontalTextPosition(JLabel.LEFT);	
\end{lstlisting}

 

  

\end{frame}

\subsubsection{Demo}
\begin{frame}
  \frametitle{Demo}
	\begin{figure}
		\includegraphics[scale=0.8]{images/jlabel.PNG}
		\caption{JLabel \\ \tiny{\textcolor{gray}{\url{http://http://java-tutorial.org/}}}}
		\end{figure}
\end{frame}


\subsection{JButton}
\begin{frame}[fragile]  %%Eine Folie
  \frametitle{JButton} %%Folientitel
  \begin{block}{Schaltfläche (Button)}
	  \begin{itemize}
		\item Vergleichbar mit einem Taster
	  \end{itemize}
  \end{block}
\begin{lstlisting}[language=java,basicstyle=\scriptsize\ttfamily]
// JButton mit Text "Drueck mich" wird erstellt
        JButton button = new JButton("Drueck mich");
 
        // JButton wird dem Panel hinzugefuegt
        panel.add(button);
\end{lstlisting}
\end{frame}

\subsubsection{Demo}
\begin{frame}
  \frametitle{Demo}
	\begin{figure}
		\includegraphics[scale=1.0]{images/jbutton.PNG}
		\caption{JButton \\ \tiny{\textcolor{gray}{\url{http://http://java-tutorial.org/}}}}
		\end{figure}
\end{frame}


\subsection{JToggleButton}
\begin{frame}[fragile, shrink]  %%Eine Folie
  \frametitle{JToggleButton} %%Folientitel
  \begin{block}{Schaltfläche, welche zwei Zustände kennt.}
	  \begin{itemize}
		\item gedrückt und nicht gedrückt
		\item Vergleichbar einem Schalter
	  \end{itemize}
  \end{block}

\begin{lstlisting}[language=java,basicstyle=\scriptsize\ttfamily]

// JToggleButton mit Text "Drueck mich" wird erstellt
        JToggleButton toggleButton = new JToggleButton("Drueck mich", true);
 
        // JToggleButton wird dem Panel hinzugefuegt
        panel.add(toggleButton);

\end{lstlisting}

\end{frame}


\subsubsection{Demo}
\begin{frame}
  \frametitle{Demo}
	\begin{figure}
		\includegraphics[scale=1.0]{images/jtogglebutton_selected.PNG}
		\caption{JToggleButton Selektiert\\ \tiny{\textcolor{gray}{\url{http://http://java-tutorial.org/}}}}
		\end{figure}
\begin{figure}
		\includegraphics[scale=1.0]{images/jtogglebutton_notselected.PNG}
		\caption{JToggleButton Nicht Selektiert \\ \tiny{\textcolor{gray}{\url{http://http://java-tutorial.org/}}}}
		\end{figure}
\end{frame}


\subsection{JCheckBox}
\begin{frame}[fragile, shrink]  %%Eine Folie
  \frametitle{JCheckBox} %%Folientitel
  \begin{block}{Auswahlkästchen, das, wenn es ausgewählt wurde, mit einem Häkchen oder Kreuz versehen wird.}
	  \begin{itemize}
		\item für Mehrfachauswahl geeignet
	  \end{itemize}
  \end{block}

\begin{lstlisting}[language=java,basicstyle=\scriptsize\ttfamily]

   //JCheckBoxen werden erstellt
        JCheckBox checkBoxMilch = new JCheckBox("Milch");
        JCheckBox checkBoxZucker = new JCheckBox("Zucker");

\end{lstlisting}

\end{frame}


\subsubsection{Demo}
\begin{frame}
  \frametitle{Demo}
	\begin{figure}
		\includegraphics[scale=0.8]{images/jcheckbox.PNG}
		\caption{JCheckBox \\ \tiny{\textcolor{gray}{\url{http://http://java-tutorial.org/}}}}
		\end{figure}
\end{frame}


\subsection{JRadioButton}
\begin{frame}[fragile, shrink]  %%Eine Folie
  \frametitle{JRadioButton} %%Folientitel
  \begin{block}{Schaltfläche zur Auswahl zwischen mehreren Optionen, in der Regel sind sie in einer ButtonGroup angeordnet. }
	  \begin{itemize}
		\item Im Gegensatz zur JCheckBox kann nur maximal eine Option selektiert werden.
	  \end{itemize}
  \end{block}

\begin{lstlisting}[language=java,basicstyle=\scriptsize\ttfamily]

 // Hier wird ein selektierter JRadioButton erstellt
        JRadioButton radio = new JRadioButton ("Ich bin ein RadioButton", true);

\end{lstlisting}

\end{frame}

\subsubsection{Demo}
\begin{frame}
  \frametitle{Demo}
	\begin{figure}
		\includegraphics[scale=0.8]{images/jradiobutton.PNG}
		\caption{JRadioButton \\ \tiny{\textcolor{gray}{\url{http://http://java-tutorial.org/}}}}
		\end{figure}
\end{frame}


\subsection{JComboBox}
\begin{frame}[fragile, shrink]  %%Eine Folie
  \frametitle{JComboBox} %%Folientitel
  \begin{block}{Dropdown-Liste (auch als Auswahlliste oder Listbox bezeichnet), die zur Auswahl eines Elementes aufgeklappt wird. }
	  \begin{itemize}
		\item Wenn die JComboBox editierbar ist, kann über ein Textfeld der ausgewählte Wert auch vom Anwender gesetzt werden.
	  \end{itemize}
  \end{block}

\begin{lstlisting}[language=java,basicstyle=\scriptsize\ttfamily]

    	// Array fuer unsere JComboBox
        String comboBoxListe[] = {"Baden-Wuerttemberg", "Bayern",
            "Berlin", "Brandenburg", "Bremen",
            "Hamburg", "Hessen", "Mecklenburg-Vorpommern",
            "Niedersachsen", "Nordrhein-Westfalen", "Rheinland-Pfalz",
            "Saarland", "Sachsen", "Sachsen-Anhalt",
            "Schleswig-Holstein", "Thueringen"};
 
        //JComboBox mit Bundeslaender-Eintraegen wird erstellt
        JComboBox bundeslandAuswahl = new JComboBox(comboBoxListe);

\end{lstlisting}

\end{frame}

\subsubsection{Demo}
\begin{frame}
  \frametitle{Demo}
	\begin{figure}
		\includegraphics[scale=0.8]{images/jcombobox.PNG}
		\caption{JComboBox \\ \tiny{\textcolor{gray}{\url{http://http://java-tutorial.org/}}}}
		\end{figure}
\end{frame}


\subsection{JList}
\begin{frame}[fragile, shrink]  %%Eine Folie
  \frametitle{JList} %%Folientitel
  \begin{block}{Einfache Liste, die mehrere Elemente enthalten kann. }
	  \begin{itemize}
		\item Einfach- und Mehrfachauswahl möglich.
	  \end{itemize}
  \end{block}

\begin{lstlisting}[language=java,basicstyle=\scriptsize\ttfamily]

 // Array fuer unsere JList
        String interessen[] = {"Politik", "Autos", "Mode", 
            "Film- und Fernsehen", "Computer", "Tiere", "Sport"};
 
        //JList mit Eintraegen wird erstellt
        JList themenAuswahl = new JList(interessen);

\end{lstlisting}

\end{frame}

\subsubsection{Demo}
\begin{frame}
  \frametitle{Demo}
	\begin{figure}
		\includegraphics[scale=0.8]{images/jlist.PNG}
		\caption{JList \\ \tiny{\textcolor{gray}{\url{http://http://java-tutorial.org/}}}}
		\end{figure}
\end{frame}



\subsection{JTextField}
\begin{frame}[fragile, shrink]  %%Eine Folie
  \frametitle{JTextField
} %%Folientitel
  \begin{block}{einfache einzeilige Texteingabe}
	  \begin{itemize}
		\item leer und vorbelegbar
	  \end{itemize}
  \end{block}

\begin{lstlisting}[language=java,basicstyle=\scriptsize\ttfamily]

// Textfeld wird erstellt
        // Text und Spaltenanzahl werden dabei direkt gesetzt
        JTextField tfName = new JTextField("Paul Programmierer", 15);
        // Schriftfarbe wird gesetzt
        tfName.setForeground(Color.BLUE);
        // Hintergrundfarbe wird gesetzt
        tfName.setBackground(Color.YELLOW);

\end{lstlisting}

\end{frame}


\subsubsection{Demo}
\begin{frame}
  \frametitle{Demo}
	\begin{figure}
		\includegraphics[scale=0.8]{images/jtextfield.PNG}
		\caption{JTextField \\ \tiny{\textcolor{gray}{\url{http://http://java-tutorial.org/}}}}
		\end{figure}
\end{frame}

\subsection{JTextArea}
\begin{frame}[fragile, shrink]  %%Eine Folie
  \frametitle{JTextArea
} %%Folientitel
  \begin{block}{einfache mehrzeilige Texteingabe}
	  \begin{itemize}
		\item wie JTextField nur für größere Eingaben
		\item Freitext
	  \end{itemize}
  \end{block}

\begin{lstlisting}[language=java,basicstyle=\scriptsize\ttfamily]

 //5-zeiliges und 20-spaltiges Textfeld wird erzeugt
        JTextArea textfeld = new JTextArea(5, 20);
 
        //Text fuer das Textfeld wird gesetzt
        textfeld.setText("Lorem ipsum dolor sit amet, " +
        		"consetetur sadipscing elitr, sed diam nonumy " +
        		"eirmod tempor invidunt ut labore et " +
        		"dolore magna aliquyam erat, sed diam voluptua. " +
        		"At vero eos et accusam et justo duo dolores et " +
                        "ea rebum.");
        //Zeilenumbruch wird eingeschaltet
        textfeld.setLineWrap(true);
 
        //Zeilenumbrueche erfolgen nur nach ganzen Woertern
        textfeld.setWrapStyleWord(true);
 
        //Ein JScrollPane, der das Textfeld beinhaltet, wird erzeugt
        JScrollPane scrollpane = new JScrollPane(textfeld); 

\end{lstlisting}

\end{frame}

\subsubsection{Demo}
\begin{frame}
  \frametitle{Demo}
	\begin{figure}
		\includegraphics[scale=1.0]{images/jtextarea.PNG}
		\caption{JTextArea \\ \tiny{\textcolor{gray}{\url{http://http://java-tutorial.org/}}}}
		\end{figure}
\end{frame}


\subsection{JScrollBar}
\begin{frame}[fragile, shrink]  %%Eine Folie
  \frametitle{JScrollBar
} %%Folientitel
  \begin{block}{Schieberregler zum Scrollen.}
	  \begin{itemize}
		\item Kann bei Platzproblemen Elemente verschieben
	  \end{itemize}
  \end{block}

\begin{lstlisting}[language=java,basicstyle=\scriptsize\ttfamily]

        //JScrollBar wird erzeugt
        JScrollBar scrollbar = new JScrollBar
                (JScrollBar.HORIZONTAL, 30, 10, 0, 100);

\end{lstlisting}

\end{frame}


\subsubsection{Demo}
\begin{frame}
  \frametitle{Demo}
	\begin{figure}
		\includegraphics[scale=1.0]{images/jscrollbar.PNG}
		\caption{JScrollBar \\ \tiny{\textcolor{gray}{\url{http://http://java-tutorial.org/}}}}
		\end{figure}
\end{frame}


\subsection{JSlider}
\begin{frame}[fragile, shrink]  %%Eine Folie
  \frametitle{JSlider
} %%Folientitel
  \begin{block}{Schieberregler, der mit einer Skala versehen werden kann.}
	  \begin{itemize}
		\item Lautstärkeregler
	  \end{itemize}
  \end{block}

\begin{lstlisting}[language=java,basicstyle=\scriptsize\ttfamily]

// JSlider-Objekt wird erzeugt
		JSlider meinSlider = new JSlider();
 
		// Mindestwert wird gesetzt
		meinSlider.setMinimum(0);
		// Maximalwert wird gesetzt
		meinSlider.setMaximum(20);
 
		// Die Abstaende zwischen den 
		// Teilmarkierungen werden festgelegt
		meinSlider.setMajorTickSpacing(5);
		meinSlider.setMinorTickSpacing(1);
 
		// Standardmarkierungen werden erzeugt 
		meinSlider.createStandardLabels(1);
 
		// Zeichnen der Markierungen wird aktiviert
		meinSlider.setPaintTicks(true);
 
		// Zeichnen der Labels wird aktiviert
		meinSlider.setPaintLabels(true);
 
		// Schiebebalken wird auf den Wert 9 gesetzt
		meinSlider.setValue(9);

\end{lstlisting}

\end{frame}

\subsubsection{Demo}
\begin{frame}
  \frametitle{Demo}
	\begin{figure}
		\includegraphics[scale=1.0]{images/jslider.PNG}
		\caption{JSlider \\ \tiny{\textcolor{gray}{\url{http://http://java-tutorial.org/}}}}
		\end{figure}
\end{frame}


\subsection{JProgressBar}
\begin{frame}[fragile, shrink]  %%Eine Folie
  \frametitle{JProgressBar} %%Folientitel
  \begin{block}{Fortschrittsbalken}
	  \begin{itemize}
		\item für Download oder Update
		\item Bearbeitung von längeren Berechnungen
	  \end{itemize}
  \end{block}

\begin{lstlisting}[language=java,basicstyle=\scriptsize\ttfamily]

// JProgressBar-Objekt wird erzeugt
		JProgressBar meinLadebalken = new JProgressBar(0, 100);
 
		// Wert fuer den Ladebalken wird gesetzt
		meinLadebalken.setValue(0);
 
		// Der aktuelle Wert wird als 
		// Text in Prozent angezeigt
		meinLadebalken.setStringPainted(true);

\end{lstlisting}

\end{frame}

\subsubsection{Demo}
\begin{frame}
  \frametitle{Demo}
	\begin{figure}
		\includegraphics[scale=1.0]{images/jprogressbar.PNG}
		\caption{JProgressBar \\ \tiny{\textcolor{gray}{\url{http://http://java-tutorial.org/}}}}
		\end{figure}
\end{frame}

\subsection{JSpinner}
\begin{frame}[fragile, shrink]  %%Eine Folie
  \frametitle{JSpinner
} %%Folientitel
  \begin{block}{Ähnlich der JComboBox, allerdings klappt die Liste nicht auf, sondern die Navigation durch die Liste erfolgt über Pfeiltasten.}
	  \begin{itemize}
		\item Warenkorb 
	  \end{itemize}
  \end{block}

\begin{lstlisting}[language=java,basicstyle=\scriptsize\ttfamily]

//JSpinner wird erzeugt
		JSpinner spinner = new JSpinner();

\end{lstlisting}

\end{frame}

\subsubsection{Demo}
\begin{frame}
  \frametitle{Demo}
	\begin{figure}
		\includegraphics[scale=1.0]{images/jspinner.PNG}
		\caption{JSpinner \\ \tiny{\textcolor{gray}{\url{http://http://java-tutorial.org/}}}}
		\end{figure}
\end{frame}

\subsection{JSeparator}
\begin{frame}[fragile, shrink]  %%Eine Folie
  \frametitle{JSeparator
} %%Folientitel
  \begin{block}{einfache Trennlinie}
	  \begin{itemize}
		\item optische Abgrenzung zwischen Blocken
	  \end{itemize}
  \end{block}

\begin{lstlisting}[language=java,basicstyle=\scriptsize\ttfamily]

        // Erzeugung eines Objektes der Klasse JSeparator
        JSeparator sep = new JSeparator();

\end{lstlisting}

\end{frame}

\subsubsection{Demo}
\begin{frame}
  \frametitle{Demo}
	\begin{figure}
		\includegraphics[scale=0.6]{images/jseparator.PNG}
		\caption{JSeparator \\ \tiny{\textcolor{gray}{\url{http://http://java-tutorial.org/}}}}
		\end{figure}
\end{frame}

\section{Anhang}
\begin{frame}
  \frametitle{Anhang} %%Folientitel
	\begin{block}{}	
		\begin{center}
			Vielen Dank für Ihre Aufmerksamkeit. \\
			Fragen?
		\end{center}	
	\end{block}
	\begin{block}{Link zur Präsentation}	
		\begin{itemize}
			\item \url{https://github.com/derdanu/akad-jav02-beamer}
		\end{itemize}
	\end{block}
\end{frame}

\subsection{Quellen}
\begin{frame} %%Eine Folie
  \frametitle{Quellen} %%Folientitel
 	\begin{itemize}
		\item \url{http://docs.oracle.com/javase/7/docs/}	
		\item \url{http://java-tutorial.org/bedienelemente.html}
	\end{itemize}
\end{frame}




\end{document}


